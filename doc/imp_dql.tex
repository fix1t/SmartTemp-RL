\section{Implementing the Deep Q-Learning Agent}

\subsection{Environment Setup}
\begin{itemize}
    \item \textbf{PyTorch Environment}: Initialize the development environment with PyTorch to ensure GPU support is enabled, enhancing computational efficiency.
    \item \textbf{Dependencies}: Import essential libraries such as \texttt{torch}, \texttt{numpy}, and \texttt{random} for numerical operations and model development.
\end{itemize}

\subsection{Architecting the DQL Agent}
\begin{itemize}
    \item \textbf{Class Definition}: Define the \texttt{DQL} class by incorporating state and action sizes, and initializing the network on the available device (GPU or CPU).
\end{itemize}

\subsection{Building the Neural Network}
\begin{itemize}
    \item \textbf{Model Architecture}: Construct the \texttt{Network} class with fully connected layers, providing details on the role of each layer from input to output. Ensure compatibility of the network architecture with state and action sizes.
\end{itemize}

\subsection{Managing Experiences with Replay Memory}
\begin{itemize}
    \item \textbf{Implementing Replay Memory}: Discuss the implementation of the \texttt{ReplayMemory} class for experience storage and sampling, highlighting its role in stabilizing learning by reducing correlation between sequential experiences.
\end{itemize}

\subsection{Executing Actions}
\begin{itemize}
    \item \textbf{Action Selection}: Explain the \texttt{act} method for action selection based on the current policy, utilizing epsilon-greedy strategy to balance exploration and exploitation.
\end{itemize}

\subsection{Training the Agent}
\begin{itemize}
    \item \textbf{Learning from Experiences}: Describe the \texttt{learn} method, focusing on the calculation of loss using the mean squared error between expected and target Q values, followed by backpropagation.
    \item \textbf{Target Network Updating}: Elaborate on the \texttt{soft\_update} method, detailing the process of gradually blending weights from the local to the target network to promote smooth learning.
\end{itemize}

\subsection{Configuring Hyperparameters}
\begin{itemize}
    \item \textbf{Tuning for Performance}: Enumerate key hyperparameters such as learning rate, minibatch size, and discount factor, offering advice on their adjustment for optimal algorithm performance.
\end{itemize}

\subsection{Compatibility with Gymnasium}
\begin{itemize}
    \item \textbf{Standardized Environment Testing}: Introduce the role of the Gymnasium interface in providing a consistent framework for testing reinforcement learning agents across standardized environments.
\end{itemize}
